\chapter{Usage}
\label{sec:usage}

\section{Introduction}
% TODO

\section{Installation}
% TODO

\subsection{Installation from binaries}
% TODO

\subsection{Installation from source tarball}
% TODO

\subsection{Installation from clone of Git repository}
% TODO

\section{BlasterSim inputs in general}
% TODO

BlasterSim uses Fortran namelist format input files.

% TODO: Discuss namelist input files in general

% TODO: Example namelist using listings package.

BlasterSim uses SI units in its input files and internally.
Support for other unit systems is not planned at the moment due to the complexity of supporting multiple unit systems.
Scientific notation can be used to appropriately scale inputs. For example, instead of 13 mm being written as \texttt{0.013} m, the user can write \texttt{13.0e-3}.

\section{Running BlasterSim}
% TODO

\section{Springer mode}
% TODO

% TODO: d_barrel
% TODO: d_plunger

\tikzmath{
\dplunger     = 3;
\lplungertube = 5;
\dbarrel      = 1;
\lbarrel      = 6;
\ldead        = 1.5;
\ylowbarrel  = (\dplunger - \dbarrel)/2; 
\yhighbarrel = \ylowbarrel + \dbarrel;
\xbarrelend   = \lplungertube + \ldead + \lbarrel;
\xbarrelstart = \lplungertube + \ldead;
\lproj      = 2;
\xplungerheadstart = 1.5;
\lplungerhead      = 0.5;
\xplungerheadend   = \xplungerheadstart + \lplungerhead;
\ycenter = \dplunger/2;
\yfullspring = -\dplunger/2;
\yfullspringbelow = -3*\dplunger/4;
\lfullspring = 7;
\ylbarrel = \ycenter + (\dplunger + \dbarrel)/4;
\xdead = \lplungertube + \ldead/2;
\xplungerheadunprimed = \lplungertube - \lplungerhead;
}

\begin{figure}
\centering
\begin{tikzpicture}
\tikzstyle{spring}=[thick,decorate,decoration={coil,amplitude=20,segment length=4pt}] % `zigzag` doesn't work right in LaTeXML, but `coil` does
\draw[thick] (\xbarrelend, \yhighbarrel) -- (\lplungertube, \yhighbarrel) -- (\lplungertube, \dplunger) -- (0, \dplunger) -- (0, 0)         -- (\lplungertube, 0) -- (\lplungertube, \ylowbarrel) -- (\xbarrelend, \ylowbarrel);
\draw[thick,dashed] (\lplungertube, \ylowbarrel) -- (\lplungertube, \yhighbarrel);
\fill[fill=gray,draw=black,thick] (\xbarrelstart, \ylowbarrel) rectangle ++(\lproj, \dbarrel);
\fill[fill=gray,draw=black,thick] (\xplungerheadstart, 0) rectangle ++(\lplungerhead, \dplunger);
\draw[spring] (0, \ycenter) -- (\xplungerheadstart, \ycenter);
\draw[<->,thick] (\xbarrelstart, \ylbarrel) -- node[above] {$l_\text{barrel}$} (\xbarrelend, \ylbarrel);
\node[draw=white] at (\xdead, \ycenter) {$V_\text{dead}$}; % `draw=white` added to make the text fully visible in LaTeXML
\draw[<->,thick] (\xplungerheadend, \ylbarrel) -- node[below] {$x_\text{p,0}$} (\lplungertube, \ylbarrel);
\end{tikzpicture}
\caption{Model spring blaster at time zero as represented in BlasterSim.}
\label{fig:springer time zero}
\end{figure}

\begin{figure}
\centering
\begin{tikzpicture}
\tikzstyle{spring_unprimed}=[thick,decorate,decoration={coil,amplitude=20,segment length=18pt}]
\tikzstyle{spring_full}=[thick,decorate,decoration={coil,amplitude=20,segment length=30pt}]
\draw[thick] (\xbarrelend, \yhighbarrel) -- (\lplungertube, \yhighbarrel) -- (\lplungertube, \dplunger) -- (0, \dplunger) -- (0, 0)         -- (\lplungertube, 0) -- (\lplungertube, \ylowbarrel) -- (\xbarrelend, \ylowbarrel);
\fill[fill=gray,draw=black,thick] (\xplungerheadunprimed, 0) rectangle ++(\lplungerhead, \dplunger);
\draw[spring_unprimed] (0, \ycenter) -- (\xplungerheadunprimed, \ycenter);
\draw[spring_full] (0, \yfullspring) -- (\lfullspring, \yfullspring); % has a long straight line segment in LaTeXML for some reason
\draw[<->,thick] (0, \yfullspringbelow) -- node[below] {$l_\text{spring}$} (\lfullspring, \yfullspringbelow);
\end{tikzpicture}
\caption{Unprimed model spring blaster as represented in BlasterSim to show spring precompression.}
\label{fig:springer unprimed}
\end{figure}

\section{Pneumatic mode}
% TODO

\section{BlasterSim internal model}
% TODO

Internally, BlasterSim simulate blasters using control volumes with arbitrary flow connections between control volumes.
This allows BlasterSim to simulates spring and pneumatic blasters with the same core simulation code. This also allows simulating more atypical blasters without major code changes.
