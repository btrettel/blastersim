% chktex-file 1

\tikzmath{
\dtube     = 1;
\ltube     = 12;
\lproj     = 2;
\xproj     = 3;
\lx        = -\dtube/2;
\yxdash    = -\dtube;
\ytubedash = 2*\dtube;
\yxcvtext  = \dtube/2;
\xbarreltext = \xproj/2;
\xatmtext    = \xproj + \lproj + 3;
}

\begin{figure}
\centering
\begin{tikzpicture}
\draw[thick] (\ltube, \dtube) -- (0, \dtube) -- (0, 0) -- (\ltube, 0);

\fill[fill=gray,draw=black,thick] (\xproj, 0) rectangle ++(\lproj, \dtube);

\draw[<->,thick]    (0, \lx)    -- node[below] {$x$} (\xproj, \lx);
\draw[thick,dashed] (0, 0)      -- (0, \yxdash);
\draw[thick,dashed] (\xproj, 0) -- (\xproj, \yxdash);

% `draw=white` added to make the text fully visible in LaTeXML
\node[draw=white] at (\xbarreltext, \yxcvtext) {barrel};
\node[draw=white] at (\xatmtext,    \yxcvtext) {atmosphere};
\end{tikzpicture}
\caption{Test case which can be solved exactly.\label{fig:exact solution}}
\end{figure}
