\chapter{Development}\label{dev}

\section{Contributing}\label{contributing}

Contributions are welcome.
Contributors can file issues and pull requests on GitHub at \url{https://github.com/btrettel/blastersim}.

BlasterSim uses \href{https://trunkbaseddevelopment.com/}{trunk based development}, in other words, there are no release branches and instead releases are noted with tags. Temporary feature branches are used as necessary. Please do not squash your commits as that can make \texttt{git bisect} work poorly.

Validation data is particularly welcome.
To contribute validation data, recreate the experimental conditions in BlasterSim and file an issue with the BlasterSim inputs, file a pull request, or contact Ben Trettel directly.

BlasterSim follows a rigorous software development process.
Tests should be developed to make sure that BlasterSim features work as intended.
All new and existing tests must pass.
Existing tests can be run with \texttt{make check} on Linux and macOS and \texttt{jom check} on Windows as discussed in \secref{verval}.
The tests themselves are in the test directory and are Fortran code.
Contributors should examine existing tests to understand the testing system.
Additionally, new features should be documented.
See the docs directory for this documentation's source code in LaTeX.

\section{Compiling BlasterSim}\label{compiling}

\subsection{Compiling from source tarball}\label{compiling-from-source-tarball}

Obtain the latest BlasterSim source code from \url{http://trettel.us/blastersim/releases/blastersim-\gittag-source.zip}.

Compiling BlasterSim requires only a modern Fortran compiler and a Make program like GNU Make on Linux and macOS or \href{https://wiki.qt.io/Jom}{jom} or \href{https://learn.microsoft.com/en-us/cpp/build/reference/nmake-reference?view=msvc-170}{NMAKE} on Windows.
Unfortunately, BlasterSim will not compile with every Fortran compiler due to the units module being too large.
gfortran is recommended, but BlasterSim is also regularly tested with LLVM flang.
The Intel Fortran Compiler unfortunately will not compile BlasterSim due to a bug in their front end~\cite{green_re_2024}.

After extracting the provided source archive, \texttt{cd} into the directory and build BlasterSim.

On Linux and macOS you can type:
\begin{lstlisting}
make BUILD=release blastersim
\end{lstlisting}

On Windows you can type (if using jom):
\begin{lstlisting}
jom BUILD=release blastersim.exe
\end{lstlisting}

Note that compilation of BlasterSim can be slow.

Once built, a BlasterSim executable should be moved to a directory on your \texttt{PATH} as discussed in \secref{installation}.

\subsection{Compiling from Git}\label{compiling-git}

Building from the Git repository requires first building genunits and geninput from Ben Trettel's \href{https://github.com/btrettel/flt/}{FLT} repository.
The requirement for a modern Fortran compiler remains the same as that from \secref{compiling-from-source-tarball}.

On Linux and macOS, clone the FLT repository and build genunits and geninput:
\begin{lstlisting}
git clone https://github.com/btrettel/flt.git
cd flt
make BUILD=release genunits geninput
\end{lstlisting}

On Windows with jom, you should replace \texttt{make genunits geninput} with \texttt{jom genunits.exe geninput.exe}.

Place \texttt{genunits} or \texttt{genunits.exe} on your \texttt{PATH}.
Now you can build BlasterSim by cloning the BlasterSim repository, \texttt{cd}ing into the clone, and following the instructions in \secref{compiling-from-source-tarball}.
To be more specific for GNU Make for Linux and macOS: % chktex 13
\begin{lstlisting}
git clone https://github.com/btrettel/blastersim.git
cd blastersim
make BUILD=release blastersim
\end{lstlisting}

On Windows with jom, you should replace
\begin{lstlisting}
make BUILD=release blastersim
\end{lstlisting}
with
\begin{lstlisting}
jom BUILD=release blastersim.exe
\end{lstlisting}

And again, a BlasterSim executable should be moved to a directory on your \texttt{PATH} as discussed in \secref{installation}.

\subsection{Compiling BlasterSim releases}\label{compiling-releases}

BlasterSim releases are static binaries so that BlasterSim has no dependencies.
The Makefiles will build static binaries if \texttt{STATIC=yes} is added to the Make command.

On Linux, run \texttt{make BUILD=release STATIC=yes blastersim} to build a release version of BlasterSim.

On Windows, run \texttt{jom BUILD=release STATIC=yes blastersim.exe} to build a release version of BlasterSim.

On macOS, it appears that static binaries can not be built.
I do not yet have a good solution to this problem, but I'm working on it.

\subsection{Compiling BlasterSim documentation}\label{compiling-docs}

BlasterSim uses LaTeX for this documentation.
All of the requirements from \secref{compiling-git} are required to compile the documentation as parts of it are auto-generated from BlasterSim tests.
A LaTeX distribution like \href{https://www.tug.org/texlive/}{TeX Live} for Linux or \href{https://miktex.org/}{MikTeX} for Windows is required to compile the documentation.
On Linux, \href{http://aspell.net/}{Aspell} is also required for spell checking.

BlasterSim's PDF documentation can be compiled by running \texttt{make docs/blastersim.pdf}.
BlasterSim's HTML documentation can be compiled by running \texttt{make docs/index.html}.

Note that these Make commands will run some BlasterSim tests to generate some numbers to print in the documentation.
This will slow down compilation of the documentation
For speed, once the documentation numbers are generated, these BlasterSim tests will not be run again when compiling the documentation unless BlasterSim's code changes.

To build the test summary HTML file docs/tests.html requires that gentesthtml be built from the \href{https://github.com/btrettel/flt/}{FLT} repository mentioned previously.
This can be compiled with \texttt{make gentesthtml} on Linux or with \texttt{jom gentesthtml.exe} on Windows.

\section{BlasterSim source code conventions}\label{conventions}
% TODO

Some code comments contain bibliographic keys referring to the BlasterSim documentation's BibTeX database.
The BibTeX database is located at docs/blastersim.bib, and the associated keys can be found there.

The control volume index for the barrel is \texttt{I\_BARREL}, which is set to 1. This convention allows BlasterSim to easily know which control volume is the barrel.

% TODO: converting between LaTeX variable names and source code variable names

\section{Road map}\label{road-map}

The following features are planned for the future:
\begin{itemize}
\item An exterior ballistics calculator.
\item Uncertainty quantification.
\item Design optimization.
\item A pressure gradient model to improve accuracy for long barrels.
\item A transonic correction model to account for potential shock waves in front of the projectile.
\item BlasterSim features are taking priority to this at the moment, however, the following are planned once BlasterSim is feature-complete:
\begin{itemize}
    \item A graphical user interface for computers.
    \item A web-based interface for informal use and use on mobile devices.
\end{itemize}
\end{itemize}

%\section{Generative AI}

%All BlasterSim code is written by a human.
%No BlasterSim code was written by generative AI.
%AI was used in BlasterSim's development only as a research and learning tool, not to write code.
