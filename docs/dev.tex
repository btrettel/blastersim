\chapter{Development}
\label{sec:dev}

\section{Compiling BlasterSim}
\label{sec:compiling}

\subsection{Compiling from source tarball}
\label{sec:compiling from source tarball}

Obtain the latest BlasterSim source code from \url{http://trettel.us/blastersim/releases/blastersim-\gittag-source.zip}.

The provided source code has no dependencies other than a modern Fortran compiler and a Make program like GNU Make on Linux and macOS or \href{https://wiki.qt.io/Jom}{jom} or \href{https://learn.microsoft.com/en-us/cpp/build/reference/nmake-reference?view=msvc-170}{NMAKE} on Windows.
Unfortunately, BlasterSim will not compile with every Fortran compiler due to the units module being too large.
gfortran is recommended, but BlasterSim is also regularly tested with LLVM flang.
\href{https://community.intel.com/t5/Intel-Fortran-Compiler/Increasing-the-size-of-the-interface-contains-stack/td-p/1594824}{The Intel Fortran Compiler unfortunately will not compile BlasterSim due to a bug in their front end}.

After extracting the provided source archive, \texttt{cd} into the directory and build BlasterSim.

On Linux and macOS you can type \texttt{make BUILD=release blastersim}.
On Windows you can type (if using jom) \texttt{jom BUILD=release blastersim.exe}

Note that compilation of BlasterSim can be slow.

Once built, a BlasterSim executable should be moved to a directory on your \texttt{PATH} as discussed in \secref{installation}.

\subsection{Compiling from clone of Git repository}
% TODO

Building from the Git repository requires first building genunits and geninput from Ben Trettel's \href{https://github.com/btrettel/flt/}{FLT} repository.
The requirement for a modern Fortran compiler remains the same as that from \secref{compiling from source tarball}.

On Linux and macOS, clone the FLT repository and build genunits and geninput:
\begin{verbatim}
git clone https://github.com/btrettel/flt.git
cd flt
make BUILD=release genunits geninput
\end{verbatim}

On Windows with jom, you should replace \texttt{make genunits geninput} with \texttt{jom genunits.exe geninput.exe}.

Place \texttt{genunits} or \texttt{genunits.exe} on your \texttt{PATH}.
Now you can build BlasterSim by cloning the BlasterSim repository, \texttt{cd}ing into the clone, and following the instructions in \secref{compiling from source tarball}.
To be more specific for GNU Make for Linux and macOS:
\begin{verbatim}
git clone https://github.com/btrettel/blastersim.git
cd blastersim
make BUILD=release blastersim
\end{verbatim}

On Windows with jom, you should replace \texttt{make BUILD=release blastersim} with \texttt{jom BUILD=release blastersim.exe}.

And again, a BlasterSim executable should be moved to a directory on your \texttt{PATH} as discussed in \secref{installation}.

\section{Contributing}
% TODO
% add tests
% all tests must pass

% TODO: Mention GitHub URL
% TODO: filing issues

\section{Compiling BlasterSim releases}
% TODO

BlasterSim releases are static binaries so that BlasterSim has no dependencies.
The Makefiles will build static binaries if \texttt{STATIC=yes} is added to the Make command.

On Linux, run \texttt{make BUILD=release STATIC=yes blastersim} to build a release version of BlasterSim.

On Windows, run \texttt{jom BUILD=release STATIC=yes blastersim.exe} to build a release version of BlasterSim.

On macOS, it appears that static binaries can not be built.
I do not yet have a good solution to this problem, but I'm working on it.

\section{Compiling BlasterSim documentation}
% TODO

BlasterSim's PDF documentation can be compiled by running \texttt{make docs/blastersim.pdf}.
BlasterSim's HTML documentation can be compiled by running \texttt{make docs/index.html}.

Note that these Make commands will run some BlasterSim tests to generate some numbers to print in the documentation.
This will slow down compilation of the documentation
For speed, once the documentation numbers are generated, these BlasterSim tests will not be run again when compiling the documentation unless BlasterSim's code changes.

%\section{Generative AI}

%All BlasterSim code is written by a human.
%No BlasterSim code was written by generative AI.
%AI was used in BlasterSim's development only as a research and learning tool, not to write code.
